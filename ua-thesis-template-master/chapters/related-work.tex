\chapter{Related Work}

This chapter provides an overview of the current state-of-the-art regarding indoor 
positioning and presence detection in indoor environments, also taking in account
how the privacy and security of the data was handled, if it was.
Driven by technological advancements and institutional needs, nearly all 
university buildings are now equipped with internet access provided through 
Wireless Access Points (APs). In the specific case study of the University of Aveiro 
\textcolor{red}{(mencionar departamento?)}, 
these APs allow for the determination of user location with a reasonable degree 
of precision, due to the need of the user to authenticate in order to use eduroam. (\textcolor{red}{(explicar o que é o eduroam?)})
Although useful for generating interesting statistics, indoor location and presence detection
rely on the processing of sensitive information. 
This information is classified as sensitive due to its high level of detail 
and its potential to uniquely identify individuals.


Example of sensitive information used for indoor location:

\begin{itemize}
    \item \textbf{Media Access Control (MAC) Address:} A persistent hardware identifier assigned to the network interface controller. 
    In Wi-Fi tracking, this serves as a static fingerprint that uniquely identifies a user's device across different sessions and locations.
    \item \textbf{Universal User (UU):} The unique user identifier (such as a student ID, employee username, or email) 
    required for network authentication.
    \item \textbf{High-Frequency Connection Logs:} Time records containing precise association and disassociation timestamps,
    these allow for the reconstruction of a user's daily routine, duration of stay, and punctuality patterns.
    \item \textbf{Spatial Hierarchy:} Detailed topological data classifying location from macro to micro levels.
    \item \textbf{Visual and Biometric Data:} In systems utilizing optical sensors, raw data includes high-resolution images 
    or video feeds capable of revealing biometric features and behavioral signatures.
\end{itemize}

The aggregation of this data creates a risk of linkability, allowing otherwise anonymous 
location traces to be cross-referenced with external records, such as class or work schedules. 
This process effectively de-anonymizes users, subjecting them to risks including surveillance 
and profiling, which constitutes a direct violation of their right to privacy.

\section{Uses for Indoor Positioning}

\subsection{Building Occupancy}

01

02

\subsection{Room Occupancy}

07

\subsection{Energy Efficiency and Saving}

03

16

\subsection{People Counting}

02


\subsection{Statistics for Comparision with Other Sensors}

04

09

\subsection{Monitoring Employee Performance}

05

\subsection{Enhancing wWrkplace Safety/Security}

05

\subsection{Optimize Space Utilization}

05

\subsection{Tracking User Behavior}

11

\subsection{Enabling Targeted Applications (p.e Smart Cleaning, Personalized Promotions)}
\subsection{Student Attendance Monitoring}

06

07

13

15

\subsection{Navigation}

18

\subsection{Development of Indoor Positioning Method}

19









\section{Dealing with User Privacy}

Alguns não usam como o (02) porque 
neste caso é um radar de sinais

\subsection{Avoids Collecting Sensitive Data}

01

\subsection{Doesn't Deal with it}

03

05

09

17

18

19

\subsection{}

04

\subsection{Anonymized Data}

06

\subsection{Ethical Clearance}

07

\subsection{Volunteer-Based Model or Opt-in Model}

11

13

15



\section{Attacks on SGX}

Pode ser interessante mencionar alguns ataques conhecidos ao SGX
Não sei se isto deveria estar aqui ou no context