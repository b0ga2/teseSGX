\chapter{Related Work}
\label{chapter:related_work}

This chapter establishes the context for the proposed work by reviewing the state-of-the-art in 
WiFi-based indoor positioning and occupancy detection. The review focuses not only on the technical 
methodologies for extracting location metrics but also on the associated privacy and security challenges.

Driven by technological advances and institutional needs, nearly all 
buildings are now equipped with internet access provided through 
\ac{ap}. These allow for the determination of user location with a reasonable degree 
of precision, due to the need of the user to authenticate to a specific AP 
in order to use the local network.

Although useful for generating interesting statistics, indoor positioning and presence detection
rely on the processing of sensitive information. 
This information is classified as sensitive due to its high level of detail 
and its potential to uniquely identify individuals.


The aggregation of this data creates a risk of linkability, allowing otherwise anonymous 
location traces to be cross-referenced with external records, such as class or work schedules. 
This process effectively de-anonymizes users, subjecting them to risks including surveillance 
and profiling, which constitutes a direct violation of their right to privacy.

FAZER DESCRIÇÂO DO QUE VAI SER ABORDADO E CONCLUSÂO (ver tese Duarte)

\section{Indoor Positioning System}

As defined by \citet{indoor_positioning_system} an Indoor Positioning System is viewed as 
the grouping of three components: positioning principles and corresponding algorithms; technologies; hardware equipment.
These are considered to have a meaningful impact on the performance of the positioning system (see Figure \ref{fig:struct_ips}).

\begin{figure}
  \includegraphics[width=\linewidth]{figs/indoor_positioning_system.jpg}
  \caption{General structure of the model of an indoor positioning system.}
  \label{fig:struct_ips}
\end{figure}


\subsection{Technologies used for Indoor Positioning System}

There are several technologies used for an Indoor Positioning System, \citet{indoor_positioning_system}
divided them in three categories, this overview has been expanded to include Optical 
and Environmental Sensors, reflecting the methodologies found in research literature,
as shown by Table \ref{table:ips_technologies_category}.

\begin{table}[htbp]
    \centering
    \caption{Indoor Positioning Categories}
    \label{table:ips_technologies_category}
    \begin{tabular}{ l  l }
        \hline
        \textbf{Category} & \textbf{Technologies} \\ \hline
        Radio Frequency (RF) & WiFi \\
                             & Bluetooth Low Energy (BLE) \\
                             & Zigbee \\
                             & Ultra-wide band (UWB) \\
                             & Radio frequency identification (RFID) \\
                             & Indoor Global Navigation Satellite System (GNSS) \\
                             & Frequency modulation radio (FM-radio) \\ \hline
        Non-Electromagnetic Waves & Ultrasound \\
                                  & Geomagnetic Waves \\ \hline
        Full-spectrum light & Range imaging \\
                            & Visible light positioning \\
                            & Laser \\ \hline
        Optical & Cameras / Computer Vision \\
                & Infrared \\ \hline
        Environmental Sensors & CO2 Sensors \\ \hline
    \end{tabular}
\end{table}

\section {Wifi for Indoor Positioning}

Although there are several technologies used for indoor positioning system, we will focus
on Wifi, since the present use case is based on Wifi. teste

In the researched literature we came across several terms related with indoor positioning, such as
positioning, location, detection, counting and tracking. MELHORAR ESTA PARTE e citar o \citet{24_ZOU2017633}
e dizer que para o meu caso olhamos para as coisas de uma maneira mais "broad" e sem tanto detalhe
para o uso especifico

Fazer ponto bonita para alguns exemplos

There are several use cases of Wifi data to calculate for indoor positioning mentioned throughout the research literature, 
such: as Wifi probe technology done by \citet{23_WANG2018495},
\ac{rssi} performed by \citet{24_ZOU2017633}, \ac{csi} researched by \citet{25_8293759}
and many other that will mentioned throughout the present chapter.

\subsection{Building Occupancy}

Estimating building occupancy is a necessary step for optimizing building operations, in this scenario
WiFi connection counts are presented as a cost-efficient proxy for occupancy when comparing to most commom methods, 
even showing a strong correlation with actual people counts as shown by \citet{09_8913341}.

% The following is regarding the article number as 01
% 

In 2020, \citet{01_ALISHAHI2021107936} proposed a framework that used machine learning models and 
statistical analysis methods to predict patterns of building occupancy based on Wifi connections.

The authors identified that more traditional methods, often suffer from latency and high maintenance costs. 
Instead, they utilized associated WiFi device counts, which refer to all devices trying, not necessarily being able to, 
authenticate with the \ac{ap}.


The implementation of this framework demonstrated positive results, achieving an 
average prediction accuracy ($R^{2}$) of 0.98 for weekdays and 0.81 for weekends.
However, \citet{01_ALISHAHI2021107936} also connoted limitations that must be acknowledged, although
the correlation between connection counts and occupancy is significant,
the prediction models require calibration with ground-truth data to account for stationary devices 
such as printers and desktops, and that the variance in the number of devices carried by each occupant should be taken in account.
This becomes more pronounced at smaller spaces, such as room-level or zone-level analysis, 
where attributing connections to a specific zone is difficult without access to granular connection details like the
ones given by \ac{rssi}.


\subsection{Room Occupancy}


\subsection{Energy Efficiency and Saving}

% The following is regarding the article number as 03
% WiFi based occupancy detection in a complex indoor space under discontinuous wireless communication: A robust filtering based on event-triggered updating

\citet{03_WANG2019228} addressed the challenge of indoor positioning and discontinuous WiFi 
communication specifically by smartphones, a phenomenon where devices suspend data transmission 
to conserve battery, causing users to disappear from conventional scanning systems. 
To overcome this, the authors developed an event-triggered updating method that estimates occupancy 
based on entering and exiting events instead of relying on continuous signal detection, 
with these events, the system estimates the device owner's location using \ac{rssi} data. 

Furthermore, the framework integrates a location filter using \ac{rssi} thresholds to discard devices 
located outside the target zone by applying a study of the mean values of the measured data in inside 
and outside positions, and a non-human filter to exclude stationary equipment, 
such as printers, via \ac{mac} address analysis.

However, the authors acknowledge significant limitations. First, the method is tailored to the specific 
WiFi operation patterns of smartphones and may not generalize to devices with different behaviors. 
Second, the reliance on \ac{rssi} thresholds restricts the system to zone-level accuracy, in a 
virtually partitioned zones without physical walls, outside devices closer to detection nodes may be 
miscalculated as inside, and thirdly the detection errors arise from user behavior, 
such as occupants carrying multiple devices or separating from their smartphones, which the authors 
suggest requires data fusion with other sensors to resolve.


16

\subsection{People Counting}

% The following is regarding the article number as 02
% Occupancy Detection and People Counting Using WiFi Passive Radar

In the 2020 \ac{ieee} Radar Conference \citet{02_9266493} presented a methodology 
for people counting based on \ac{pwr}. The authors presented this approach 
to overcome the limitations of standard WiFi sensing, since \ac{rssi} methods are 
prone to unpredictable fluctuations and false positives due to multipath effects, 
and \ac{csi} techniques typically require specific hardware modifications 
or high-rate transmissions that degrade network throughput.

The overall results presented a high accuracy of 99.54\% for 
tasks of occupancy detection and 98.80\% for people counting. 
Taking into account the positive results, the authors defend that the \ac{pwr} 
system is applicable as it leverages existing commercial WiFi 
\ac{ap}s without requiring any modifications to the WiFi infrastructure or 
additional devices on the network. 
However, this advantage comes with added computational 
complexity, as the system relies completely on signal processing.

% The following is regarding the article number as 14_02
% Localization and Counting of Indoor Populations on a University Campus using Wi-Fi Connection Logs and Floor Plans

In 2022, for a Master's thesis, \citet{14_02_Lannois_Carroll-Woolery} proposed the "Building Floor Zone" technique to improve WiFi counting precision without 
requiring new hardware. 
The methodology involved mapping \ac{ap}s to specific zones based on room numbering and assigning hallway \ac{ap}s to the center of adjacent rooms.
Experimental analysis confirmed that this approach resulted in a higher statistical correlation with schedule based ground truth compared to standard floor level aggregation. 
An interesting mention is that the study found that counting all unique connections provided better accuracy than filtering for long-duration sessions, challenging the assumption that 
transient users are merely noise. 

\subsection{Statistics for Comparision with Other Sensors}

% The following is regarding the article number as 04
% Effectiveness of using WiFi technologies to detect and predict building occupancy

In 2017, \citet{04_OUF2017005} presented a case study to demonstrate the effectiveness of using 
WiFi to detect occupancy as opposed to the more common $CO_2$ sensors.
To facilitate this comparison, the authors simultaneously monitored WiFi connection counts and 
$CO_2$ concentration levels in a university classroom over one week, using manual occupant counts 
as ground truth.

The analysis revealed that WiFi counts served as a superior predictor of occupancy, 
exhibiting a stronger statistical correlation ($r=0.839$) than $CO_2$ levels ($r=0.728$).
Furthermore, the authors highlighted that while $CO_2$ sensors suffered from a detection lag of 
approximately 20 minutes and susceptibility to non-occupancy related fluctuations, WiFi data 
provided a more accurate, real-time proxy for occupancy with the added benefit of utilizing existing 
infrastructure without additional cost.

% The following is regarding the article number as 09
% Role of Campus WiFi Infrastructure for Occupancy Monitoring in a Large University

In the 2018 IEEE International Conference on Information and Automation for Sustainability (ICIAfS), \citet{09_8913341} 
proposed a study to assess the feasibility of using WiFi \ac{ap} infrastructure for room-level occupancy monitoring across the University of New South Wales campus, a scenario similar to this thesis. 
The study had a duration of four weeks across rooms with varying numbers of \ac{ap}s, comparing WiFi data against beam counter sensors and ground truth enrollment numbers. 
To account for transient users, passing by, connecting to the \ac{ap}s, connections lasting less than five minutes were filtered out.
The results indicated that the WiFi method achieved a stronger correlation with actual occupancy ($R=0.85$) compared to the beam counters ($R=0.68$). 
it also demonstrated a lower error rate, with a \ac{smape} of 12.1\% compared to 15.6\% for the hardware sensors. 


\subsection{Tracking User Behavior}

% The following is regarding the article number as 05
% TRACKING INDOOR LOCATION , MOVEMENT AND DESK OCCUPANCY IN THE WORKPLACE

A case study performed by \citet{05_CRACKED_LABS} in 2024 analyzed the privacy implications of existing technologies for behavioral monitoring and profiling. 
The report examined solutions from major vendors such as Cisco, Juniper, Spacewell, and Locatee, identifying a trend where 
workplace infrastructure is used for tracking user/employee behavior.
Specifically in Section 3.2, the study focuses on Cisco, the manufacturer of the devices generating the logs utilized in this thesis. 
The report highlights a product called Cisco Spaces, a cloud-based platform that processes massive amounts of data
to profile user behavior. Regarding the methodology of user classification, the report notes that the system categorizes 
located persons into groups such as employee, student or guest based on the SSID category. 
This suggests that the system identifies the type of user based on the specific WiFi network they connect to, allowing for distinct tracking of 
different user groups.

% The following is regarding the article number as 10
% EDUM: Classroom Education Measurements via Large-scale WiFi Networks

In 2016, \citet{10_10.1145/2971648.2971657} proposed the EDUM (EDUcation Measurement) system to characterize educational behaviors using data collected from large-scale WiFi infrastructures, 
analyzing students punctuality based on longitudinal WiFi connection traces and to assess lecture attractiveness and student distraction.
The system utilizes the mobile phone's interactive state, the screen on/off status, at a per-minute basis.
Deployed at Tsinghua University with approximately 700 student volunteers and 2,800 \ac{ap}s, the study provided results that revealed a negative correlation between high mobile phone usage during 
class and academic performance (GPA), and confirmed that students seated in the back rows exhibit significantly higher distraction levels than those in the front.


11
% The following is regarding the article number as 11
% Integrating Indoor Localization Technologies for Enhanced Smart Education: Challenges, Innovations, and Applications

Later in 2025, \citet{11_11030449} conducted a study to investigate indoor location technologies for future integration in Smart Education (SE) environments, the authors 
reviewed several technologies, as mentioned earlier on this chapter, but for the sake of the thesis the focus will be toward WiFi. 
The authors identified WiFi as the most accessible technology for SE, citing its widespread availability in educational institutions and the seamless 
connectivity provided by networks like eduroam, the one used on the present thesis.

Regarding its capabilities, the study emphasized the potential of the IEEE 802.11mc standard (Fine Time Measurement), which allows for precise 
ranging, approximately 1 meter, and preserves user privacy with calculations on the client device side. 
However, in their experimental Proof of Concept (POC) for automatic attendance, the WiFi only approach achieved a classification accuracy of 93.77\% using a regression model. 
While effective, it was slightly outperformed by 5G (97.21\%), leading the authors to conclude that while WiFi is a useful 
standalone tool due to its low cost and presence, a fusion of technologies provides the most robust solution for critical attendance monitoring.

\subsection{Enabling Targeted Applications (p.e Smart Cleaning, Personalized Promotions)}


\subsection{Student Attendance Monitoring}

% The following is regarding the article number as 06 and 07 (precursor)
% Modeling Classroom Occupancy Using Data of WiFi Infrastructure in a University Campus

In 2022, \citet{06_9750047} published in the IEEE Sensors Journal a machine learning framework to infer classroom occupancy, 
sharing the same goal as the current work but employing standard statistical modeling. 
The authors utilized daily WiFi logs from the university's IT department, a scenario identical to this thesis, comprising 
data from 70 \ac{ap}s including \ac{uu}, \ac{mac} address, event timestamps, and \ac{ap} names.
Besides the similar data source, \citet{06_9750047} identified critical limitations in using raw logs, 
such as the overlapping coverage of \ac{ap}s (where a student inside a room connects to a hallway \ac{ap}), 
the presence of "bystanders", and the variability of multiple device connections per user.
Ultimately, the framework achieved a symmetric Mean Absolute Percentage Error of 13.1\%, a result comparable to dedicated beam-counter sensors, 
this once more demonstrated that existing WiFi infrastructure can yield accurate occupancy estimation 
with no additional deployment costs only computing costs.

% The following is regarding the article number as 08
% Attendance monitoring in classroom using smartphone & Wi-Fi fingerprinting
Earlier in 2016, \citet{08_7814796} proposed a mobile attendance system that combines facial recognition for user authentication 
with WiFi network analysis to verify the student's location. 
While the authors acknowledged methods like trilateration 
\footnote{Trilateration is a geometric method of determining location by measuring distances from at least three known reference points using signal strength.}, 
but due to signal interference adopted a fingerprinting approach using the k-Nearest Neighbor algorithm.
Regarding the results, the system demonstrated high precision, achieving a positioning error between 1 and 2.5 meters. 
In practical testing, this methodology identified whether a student was inside the target classroom 94\% of the time, 
proving that low-cost fingerprinting can effectively predict proxy attendance without the need for dedicated hardware.


13
% The following is regarding the article number as 13
% Classroom Occupancy-based Human Resource Optimization using Sensor- and WiFi-based Location Tracking

In a Master's thesis, \citet{13_article} developed a framework for optimizing human resource allocation, directed for cleaning and maintenance, 
this was done by comparing three occupancy detection models: static university course schedules, thermal occupancy sensors, and WiFi location tracking. 
The study highlighted that while schedule based models are often inaccurate due to student absenteeism, thermal sensors provided the highest accuracy of approximately 
98\% by counting heat signatures at doorways.

However, the author emphasized the prohibitive scalability of thermal sensors, due to a large installation cost. 
In contrast, the WiFi based model, utilizing the existing network achieved a comparable accuracy of roughly 90\% with zero additional 
infrastructure costs. 
A supporting survey within the study validated the viability of this approach, revealing that 95.2\% of campus users carried smartphones, 
with 90.5\% actively logged into the university network, confirming that WiFi connection counts serve as a reliable, cost-effective proxy for real-time occupancy.

15

\subsection{Navigation}



\subsection{Development of Indoor Positioning Method}

19









\section{Dealing with User Privacy}

Alguns não usam como o (02) porque 
neste caso é um radar de sinais

\subsection{Avoids Collecting Sensitive Data}

01

\subsection{Doesn't Deal with it}

03

05

09

17

18

19

\subsection{}

04

\subsection{Anonymized Data}

06

\subsection{Ethical Clearance}

07

\subsection{Volunteer-Based Model or Opt-in Model}

11

13

15

