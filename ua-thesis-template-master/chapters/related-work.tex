\chapter{Related Work}
\label{chapter:related_work}

This chapter establishes the context for the proposed work by reviewing the state-of-the-art in WiFi based indoor positioning and occupancy detection. 
While nearly identical solutions, combining already existing infrastructure monitoring with hardware enforced privacy, are scarce in the literature, 
distinct bodies of research provide the necessary pillars for this thesis.
The review begins by analyzing technical methodologies for Indoor Positioning using standard WiFi technologies, distinguishing between active localization 
and passive occupancy estimation. 
It then transitions to the critical challenge of User Privacy in location analytics. 
Finally, the chapter concludes with an analysis of \ac{SGX} as a possibility for secure indoor positioning data processing, evaluating its current status and 
performance in privacy preserving applications.

\section{Indoor Positioning System}

As defined by \citet{indoor_positioning_system} an Indoor Positioning System is viewed as 
the grouping of three components: positioning principles and corresponding algorithms, technologies and hardware equipment.
These are considered to have a meaningful impact on the performance of the positioning system (see Figure \ref{fig:struct_ips}).

\begin{figure}
  \includegraphics[width=\linewidth]{figs/indoor_positioning_system.jpg}
  \caption{General structure of the model of an indoor positioning system by \citet{indoor_positioning_system}.}
  \label{fig:struct_ips}
\end{figure}


\subsection{Technologies used for Indoor Positioning System}

There are several technologies used for an Indoor Positioning System, \citet{indoor_positioning_system}
divided them in three categories, this overview has been expanded to include Optical 
and Environmental Sensors, reflecting the methodologies found in research literature,
as shown by Table \ref{table:ips_technologies_category}.

\begin{table}[htbp]
    \centering
    \caption{Indoor Positioning Categories}
    \label{table:ips_technologies_category}
    \begin{tabular}{ l  l }
        \hline
        \textbf{Category} & \textbf{Technologies} \\ \hline
        Radio Frequency (RF) & WiFi \\
                             & Bluetooth Low Energy (BLE) \\
                             & Zigbee \\
                             & Ultra-wide band (UWB) \\
                             & Radio frequency identification (RFID) \\
                             & Indoor Global Navigation Satellite System (GNSS) \\
                             & Frequency modulation radio (FM-radio) \\ \hline
        Non-Electromagnetic Waves & Ultrasound \\
                                  & Geomagnetic Waves \\ \hline
        Full-spectrum light & Range imaging \\
                            & Visible light positioning \\
                            & Laser \\ \hline
        Optical & Cameras / Computer Vision \\
                & Infrared \\ \hline
        Environmental Sensors & CO2 Sensors \\ \hline
    \end{tabular}
\end{table}

\section {Wifi for Indoor Positioning}

In the researched literature, several terms related to indoor positioning appear interchangeably, such as positioning, localization, detection, counting, and tracking. 
While authors like \citet{24_ZOU2017633} often distinguish these terms based on technical granularity, where localization implies pinpointing specific $(x,y)$ coordinates 
and tracking involves temporal notion, this thesis adopts a broader operational perspective. 

Instead of focusing on the precise coordinate-level tracking of individual targets, the current work prioritizes occupancy detection and crowd estimation in a certain area, 
in this context, the specific terminology is less about the geometric precision of a single user's location and more about the aggregate presence of devices within a defined zone, 
treating location as a categorical state.

To achieve these various levels of granularity, ranging from simple presence detection to precise tracking, researchers leverage different characteristics of the WiFi signal. 
Consequently, the literature presents several distinct methodologies for indoor positioning, each utilizing a specific data metric. 
Prominent examples include the analysis of WiFi probe requests as demonstrated by \citet{23_WANG2018495}, 
the measurement of \ac{rssi} performed by \citet{24_ZOU2017633}, and the more complex \ac{csi} techniques researched by \citet{25_8293759}. 
These approaches, along with others, form the technical basis for the studies reviewed throughout this chapter.

The following chapter will be divided according to the main goal of the Wifi for Indoor Positioning.


\subsection{Energy Savings and Efficiency}

Estimating building occupancy is a necessary step for optimizing building operations, in this scenario
WiFi connection counts are presented as a cost-efficient proxy for occupancy when comparing to most commom methods, 
even showing a strong correlation with actual people counts as shown by \citet{09_8913341}.

% The following is regarding the article number as 01
% A framework to identify key occupancy indicators for optimizing building operation using WiFi connection count data
In 2021, \citet{01_ALISHAHI2021107936} proposed a framework that utilizes machine learning and statistical analysis to extract key occupancy indicators from simple WiFi 
connection counts. The authors argued that traditional hardware sensors often suffer from latency, high maintenance costs, and limited scalability. 
In contrast, they leveraged associated WiFi device counts, defined as devices attempting to authenticate with the \ac{ap}, as a cost-effective proxy for human presence.
The framework yielded strong results, achieving an average prediction accuracy ($R^{2}$) of 0.98 for weekdays and 0.81 for weekends,
validating the reliability of WiFi logs for occupancy estimation. 
However, \citet{01_ALISHAHI2021107936} also highlighted critical limitations that directly impact the design of this thesis's methodology. 
They noted that raw connection counts are noisy and require calibration with ground-truth data to account for two main sources of error, the stationary devices like 
printers, desktops, and projectors that permanently connect to the network must be filtered out and the device to occupant ratio, which is the variance in the number of 
devices carried by each user. The study emphasizes that these errors become more pronounced at smaller spaces, where attributing a connection to a specific zone is 
difficult without the precise localization data provided by metrics like \ac{rssi}.

% The following is regarding the article number as 02
% Occupancy Detection and People Counting Using WiFi Passive Radar
In the 2020 \ac{ieee} Radar Conference \citet{02_9266493} presented a methodology 
for people counting based on \ac{pwr}. The authors presented this approach 
to overcome the limitations of standard WiFi sensing, since \ac{rssi} methods are 
prone to unpredictable fluctuations and false positives due to multipath effects, 
and \ac{csi} techniques typically require specific hardware modifications 
or high-rate transmissions that degrade network throughput. 
The overall results presented a high accuracy of 99.54\% for 
tasks of occupancy detection and 98.80\% for people counting. 
Taking into account the positive results, the authors defend that the \ac{pwr} 
system is applicable as it leverages existing commercial WiFi 
\ac{ap}s without requiring any modifications to the WiFi infrastructure or 
additional devices on the network. 
However, this advantage comes with added computational 
complexity, as the system relies completely on signal processing.

% The following is regarding the article number as 03
% WiFi based occupancy detection in a complex indoor space under discontinuous wireless communication: A robust filtering based on event-triggered updating
\citet{03_WANG2019228} addressed the critical challenge of discontinuous communication, where smartphones suspend WiFi transmission to conserve battery, 
causing users to intermittently disappear from network scans. To maintain accuracy during these silence periods, the authors developed an event-triggered updating 
framework that estimates occupancy based on entry and exit events rather than relying on continuous signal presence. 
Furthermore, the framework integrates a location filter using \ac{rssi} thresholds to discard devices 
located outside the target zone by applying a study of the mean values of the measured data in inside 
and outside positions which could be helpful for the current work, and a non-human filter to exclude stationary equipment, 
such as printers, via \ac{mac} address analysis. 
The authors also acknowledge significant limitations such as the reliance on \ac{rssi} thresholds restricts the system to zone-level accuracy, the behavior 
of smartphones suspension of data transmission and the detection errors arise from user behavior, 
such as occupants carrying multiple devices or separating from their smartphones.

% The following is regarding the article number as 04
% Effectiveness of using WiFi technologies to detect and predict building occupancy
In 2017, \citet{04_OUF2017005} presented a case study to demonstrate the effectiveness of using 
WiFi to detect occupancy as opposed to the more common $CO_2$ sensors. 
To facilitate this comparison, the authors simultaneously monitored WiFi connection counts and 
$CO_2$ concentration levels in a university classroom over one week, using manual occupant counts 
as ground truth. 
The analysis revealed that WiFi counts served as a superior predictor of occupancy, 
exhibiting a stronger statistical correlation ($r=0.839$) than $CO_2$ levels ($r=0.728$).
Furthermore, the authors highlighted that while $CO_2$ sensors suffered from a detection lag of 
approximately 20 minutes and susceptibility to non-occupancy related fluctuations, WiFi data 
provided a more accurate, real-time proxy for occupancy with the added benefit of utilizing existing 
infrastructure without additional cost.

% The following is regarding the article number as 13
% Classroom Occupancy-based Human Resource Optimization using Sensor- and WiFi-based Location Tracking
In a Master's thesis, \citet{13_thesis} developed a framework for optimizing human resource allocation, directed for cleaning and maintenance, 
this was done by comparing three occupancy detection models: static university course schedules, thermal occupancy sensors, and WiFi location tracking. 
The study highlighted that while schedule based models are often inaccurate due to student absenteeism, thermal sensors provided the highest accuracy of approximately 
98\% by counting heat signatures at doorways. 
However, the author emphasized the negative considerations on the scalability of thermal sensors, due to a large installation cost. 
In contrast, the WiFi based model, utilizing the existing network achieved a comparable accuracy of roughly 90\% with zero additional 
infrastructure costs. 
A supporting survey within the study validated the viability of this approach, revealing that 95.2\% of campus users carried smartphones, 
with 90.5\% actively logged into the university network, confirming that WiFi connection counts serve as a reliable, cost-effective proxy for real-time occupancy.

% The following is regarding the article number as 16
% Classroom Automation Using RSSI
In 2019, \citet{16_8938387} proposed an architecture model for an occupancy monitoring system based on the \ac{rssi} detected by low-cost ESP8266 microcontrollers. 
To address the inherent instability of wireless signals, the system employs a mean algorithm, which triggers an occupancy state only when the average \ac{rssi} over 
seven iterations dips below a calibrated threshold. 
Experimental trials demonstrated a high detection accuracy of 95\%, approximately one error every 20 minutes, but only within a limited 3-meter line-of-sight range. 
The authors explicitly noted that beyond this 3-meter radius, the accuracy degrades significantly due to multipath propagation. 
\footnote{Phenomenon in wireless communication where a transmitted signal reaches the receiver antenna through multiple paths due to reflections and scattering 
from environmental objects, resulting in different signal copies with varying attenuation, phase shifts, and time delays as defined by \citet{KALUUBA2006621}}
This observation is particularly relevant to the current thesis, as it highlights the physical limitations of \ac{rssi} based ranging in complex indoor environments, 
confirming that simple signal strength thresholds are often insufficient for reliable room-level tracking in larger educational spaces.

\subsection{Behavioral Monitoring}

% The following is regarding the article number as 05
% TRACKING INDOOR LOCATION , MOVEMENT AND DESK OCCUPANCY IN THE WORKPLACE
A case study performed by \citet{05_CRACKED_LABS} in 2024 analyzed the privacy implications of existing technologies for behavioral monitoring and profiling. 
The report examined solutions from major vendors such as Cisco, Juniper, Spacewell, and Locatee, identifying a trend where 
workplace infrastructure is used for tracking user/employee behavior. 
Specifically in Section 3.2, the study focuses on Cisco, the manufacturer of the devices generating the logs utilized in this thesis. 
The report highlights a product called Cisco Spaces, a cloud-based platform that processes massive amounts of data
to profile user behavior. Regarding the methodology of user classification, the report notes that the system categorizes 
located persons into groups such as employee, student or guest based on the SSID category. 
This suggests that the system identifies the type of user based on the specific WiFi network they connect to, allowing for distinct tracking of 
different user groups.

% The following is regarding the article number as 10
% EDUM: Classroom Education Measurements via Large-scale WiFi Networks
In 2016, \citet{10_10.1145/2971648.2971657} proposed the EDUM (EDUcation Measurement) system to characterize educational behaviors using data collected from large-scale WiFi infrastructures, 
analyzing students punctuality based on longitudinal WiFi connection traces and to assess lecture attractiveness and student distraction.
The system utilizes the mobile phone's interactive state, the screen on/off status, at a per-minute basis.
Deployed at Tsinghua University with approximately 700 student volunteers and 2,800 \ac{ap}s, the study provided results that revealed a negative correlation between high mobile phone usage during 
class and academic performance (GPA), and confirmed that students seated in the back rows exhibit significantly higher distraction levels than those in the front.

% The following is regarding the article number as 11
% Integrating Indoor Localization Technologies for Enhanced Smart Education: Challenges, Innovations, and Applications
Later in 2025, \citet{11_11030449} conducted a study to investigate indoor location technologies for future integration in Smart Education (SE) environments, the authors 
reviewed several technologies, as mentioned earlier on this chapter, but for the sake of the thesis the focus will be toward WiFi. 
The authors identified WiFi as the most accessible technology for SE, citing it's widespread availability in educational institutions and the seamless 
connectivity provided by networks like eduroam, the one used on the present thesis. 
Regarding it's capabilities, the study emphasized the potential of the IEEE 802.11mc standard (Fine Time Measurement), which allows for precise 
ranging, approximately 1 meter, and preserves user privacy with calculations on the client device side. 
However, in their experimental Proof of Concept (POC) for automatic attendance, the WiFi only approach achieved a classification accuracy of 93.77\% using a regression model. 
While effective, it was slightly outperformed by 5G (97.21\%), leading the authors to conclude that while WiFi is a useful 
standalone tool due to it's low cost and presence, a fusion of technologies provides the most robust solution for critical attendance monitoring.

\subsection{Occupancy Monitoring}

% The following is regarding the article number as 06 and 07 (precursor)
% Modeling Classroom Occupancy Using Data of WiFi Infrastructure in a University Campus
In 2022, \citet{06_9750047} published in the IEEE Sensors Journal a machine learning framework to infer classroom occupancy, 
sharing the same goal as the current work but employing standard statistical modeling. 
The authors utilized daily WiFi logs from the university's IT department, a scenario identical to this thesis, comprising 
data from 70 \ac{ap}s including \ac{uu}, \ac{mac} address, event timestamps, and \ac{ap} names. 
Besides the similar data source, \citet{06_9750047} identified critical limitations in using raw logs, 
such as the overlapping coverage of \ac{ap}s (where a student inside a room connects to a hallway \ac{ap}), 
the presence of "bystanders", and the variability of multiple device connections per user. 
Ultimately, the framework achieved a symmetric Mean Absolute Percentage Error of 13.1\%, a result comparable to dedicated beam-counter sensors, 
this once more demonstrated that existing WiFi infrastructure can yield accurate occupancy estimation 
with no additional deployment costs only computing costs.

% The following is regarding the article number as 09
% Role of Campus WiFi Infrastructure for Occupancy Monitoring in a Large University
In the 2018 IEEE International Conference on Information and Automation for Sustainability (ICIAfS), \citet{09_8913341} 
proposed a study to assess the feasibility of using WiFi \ac{ap} infrastructure for room-level occupancy monitoring across the University of New South Wales campus, a scenario similar to this thesis. 
The study had a duration of four weeks across rooms with varying numbers of \ac{ap}s, comparing WiFi data against beam counter sensors and ground truth enrollment numbers. 
To account for transient users, passing by, connecting to the \ac{ap}s, connections lasting less than five minutes were filtered out. 
The results indicated that the WiFi method achieved a stronger correlation with actual occupancy ($R=0.85$) compared to the beam counters ($R=0.68$). 
it also demonstrated a lower error rate, with a \ac{smape} of 12.1\% compared to 15.6\% for the hardware sensors. 

% The following is regarding the article number as 14_02
% Localization and Counting of Indoor Populations on a University Campus using Wi-Fi Connection Logs and Floor Plans
In 2022, for a Master's thesis, \citet{14_02_thesis} proposed the "Building Floor Zone" technique to improve WiFi counting precision without requiring new hardware. 
The methodology involved mapping \ac{ap}s to specific zones based on room numbering and assigning hallway \ac{ap}s to the center of adjacent rooms, in similarity with the present thesis. 
Experimental analysis confirmed that this approach resulted in a higher statistical correlation with schedule based ground truth compared to standard floor level aggregation. 
An interesting mention is that the study found that counting all unique connections provided better accuracy than filtering for long-duration sessions, challenging the assumption that 
transient users are merely noise. 

% The following is regarding the article number as 08
% Attendance monitoring in classroom using smartphone & Wi-Fi fingerprinting
Earlier in 2016, \citet{08_7814796} proposed a mobile attendance system that combines facial recognition for user authentication 
with WiFi network analysis to verify the student's location. 
While the authors acknowledged methods like trilateration 
\footnote{Trilateration is a geometric method of determining location by measuring distances from at least three known reference points using signal strength.}, 
but due to signal interference adopted a fingerprinting approach using the k-Nearest Neighbor algorithm. 
Regarding the results, the system demonstrated high precision, achieving a positioning error between 1 and 2.5 meters. 
In practical testing, this methodology identified whether a student was inside the target classroom 94\% of the time, 
proving that low-cost fingerprinting can effectively predict proxy attendance without the need for dedicated hardware.

% The following is regarding the article number as 14
% WiFi Received Signal Strength (RSS) Based Automated Attendance System for Educational Institutions
\citet{14_10.1145/3704522.3704523} presented a smartphone based attendance system utilizing WiFi signal strength for indoor localization. 
As noted in previous works, distinguishing user presence within confined boundaries, like a specific room, is challenging due to signal fluctuations. 
To address this, the authors employed a zone based fingerprinting approach, dividing the space into virtual grids and using machine learning to classify them. 
A valuable insight for this thesis is the authors explicit classification of signal quality, which categorizes \ac{rssi} values into ranges according to signal strength, 
something that could be necessary in the current work. 
The authors system achieved 100\% accuracy in one test room and 92\% in another. 
Furthermore, the study demonstrated significant robustness: the accuracy remained stable even when individual \ac{ap}s were removed or 
when different smartphone models were used.

\subsection{Fingerprinting}
17
% The following is regarding the article number as 18
% WIFI FINGERPRINT INDOOR POSITIONING SYSTEM USING PROBABILITY DISTRIBUTION COMPARISON
In a study focusing on probabilistic localization, \citet{18_6288374} developed a WiFi fingerprinting system that utilizes probability distribution comparisons 
rather than simple \ac{rssi} averaging. Crucially for the context of this thesis, the methodology also used a offline phase to construct a radio map, 
a database where fingerprints were manually collected at several known rooms throughout the building. 
During the offline phase, the authors collected 100 \ac{rssi} samples at each reference point to estimate the signal strength probability 
distributions for every visible access point. This data collection process demonstrates the significant calibration effort, necessary in fingerprinting approaches. 
While their method achieved a median positioning error of 2.4 meters, the requirement to build and maintain such a detailed radio map highlights the scalability challenges 
that the passive, infrastructure based approach proposed in this current work seeks to eliminate.

% The following is regarding the article number as 19
% WIFI FINGERPRINT INDOOR POSITIONING SYSTEM USING PROBABILITY DISTRIBUTION COMPARISON
Similarly, in 2020, \citet{19_NINH2020238} proposed a random statistical method for indoor positioning that relies on the same two phase architecture common to 
fingerprinting solutions. The system explicitly distinguishes between an offline handling process, where a large number of WiFi signals are collected at specific 
reference points to create a database, and an online positioning process that uses the Mahalanobis distance to match live user data against this stored radio map.

\section{Approaches to Privacy in Literature}

The ubiquity of WiFi networks as a sensing infrastructure presents a fundamental problem: the same granular data that enables precise occupancy monitoring 
also introduces significant privacy risks for the individuals being tracked. As noted in the previous sections, transforming a standard wireless network into a 
soft sensor involves analyzing device footprints, specifically \ac{mac} addresses and signal characteristics, which can serve as strong identifiers of specific individuals.

Consequently, the academic literature reveals a spectrum of methodologies designed to reconcile the trade-off between service utility  and user privacy. 
These approaches range from simple data minimization strategies, which avoid collecting unique identifiers entirely, to complex anonymization schemes like hashing 
and k-anonymity. However, as the demand for real-time, room-level precision increases, so does the inadequacy of traditional obfuscation methods. 

This section reviews these divergent privacy-preserving strategies, categorizing them into distinct paradigms found in the literature: 
omission of privacy mechanisms, where technical feasibility takes precedence; 
data aggregation and minimization, where sensitive data is aggregated at the source; 
anonymization and pseudonymization, where identifiers are masked; 
institutional compliance and ethical clearance, relying on governance frameworks; 
and volunteer-based or opt-in models. 

The chapter concludes by examining hardware enforced security, specifically focusing on Trusted Execution Environments like \ac{sgx}, which represent the state-of-the-art in 
secure location analytics.

\subsection{Omission/Inexistence of Privacy Mechanisms}

\citet{03_WANG2019228} focused primarily on the technical feasibility of detecting inactive smartphones to improve zone detection accuracy. 
While the system utilizes \ac{mac} address analysis to filter out non-human devices, the study does not explicitly detail mechanisms for anonymizing these addresses or 
protecting the identity of the smartphone owners.

Furthermore, \citet{05_CRACKED_LABS} provided an analysis of commercial indoor positioning systems, identifying a lack of 
privacy-by-design. The report highlights that these platforms often prioritize behavioral profiling, categorizing users into groups like employees or 
students, over data protection, allowing for the tracking of individual movements without obfuscation or user optional mechanisms.

\citet{09_8913341} conducted a large scale validation of WiFi occupancy sensing by benchmarking it against hardware beam counters across a university campus. 
The study focused almost exclusively on the utility of the metadata, like user's \ac{mac} address which can endanger students privacy, since 
the authors did not detail privacy preserving architectures, it implies the inexistence of privacy mechanisms.

\subsection{Data Aggregation and Minimization}

\citet{01_ALISHAHI2021107936} adopted a data minimization strategy by relying only on aggregated connection counts per \ac{ap}. 
By converting raw network logs into numerical totals before analysis, the framework inherently discards individual identifiers, 
ensuring that no unique user data is retained.

\citet{02_9266493} took the concept of data minimization to the physical layer by employing a device free approach known as Passive WiFi Radar. 
Instead of decoding data packets or logging \ac{mac} addresses, this method analyzes signal reflections caused by moving bodies. 
Consequently, the system avoids collecting any digital identifiers, making the data inherently anonymous and decoupling the occupancy detection from the users' 
personal devices.

Similarly, \citet{04_OUF2017005} employed an aggregation strategy to validate WiFi sensing against environmental benchmarks. 
By utilizing the total number of connections as a bulk metric to correlate with $CO_2$ levels, the authors treated the crowd as a single entity. 
This approach avoids the privacy difficulties of trajectory tracking, as the system monitors the state of the room and not the behavior of the individuals within it.

\citet{13_thesis} demonstrated a resource optimization framework that relies on aggregated occupancy density. 
By only using bulk connection counts directly from the university's central IT department, rather than logging individual devices, the system calculates the "usage intensity"  
of classrooms.

\subsection{Anonymization and Pseudonymization}

\citet{14_Carroll_Woolery} implemented a form of pseudonymization to mitigate the risks of long-term profiling. 
In this study, the unique identifiers were masked using a hashing algorithm that was reset every 24 hours.


06

\subsection{Institutional Compliance and Ethical Clearance}

\citet{06_9750047} addressed the privacy implications of tracking students by explicitly operating under a governance framework. 
The study obtained formal ethical clearance from the university.


\subsection{Volunteer-Based Model or Opt-in Model}

\citet{08_7814796} implemented a privacy model based on user interaction. In their system, attendance is not monitored passively, instead, students must actively engage 
with a smartphone application to capture a facial scan and submit their WiFi fingerprint, which falls on the category of an opt-in, ensuring that location data is only 
transmitted with the user's direct consent and knowledge.

\citet{10_10.1145/2971648.2971657} employed a volunteer based model to justify the collection of highly granular behavioral data. In their "EDUM" system, 
the researchers recruited approximately 700 students who consented to having their WiFi traces and mobile application usage monitored. 

\citet{11_11030449} highlighted the privacy advantages of the IEEE 802.11mc standard, which shifts the location calculation to the client device. 
To demonstrate this architecture, the authors developed a mobile application, by requiring students to actively install and engage with this software to register their 
presence, the system uses a opt-in model.

Likewise, \citet{14_3704522.3704523} developed a smartphone based attendance system that leverages \ac{rssi} zoning to verify student presence. 
Since the primary utility is the validation of attendance records using the students smartphone, the architecture operates on an opt-in model where 
students consent to the tracking of their device's location state in exchange for the administrative benefit of automated attendance logging.

Similarly, \citet{14_3704522.3704523} proposed a smartphone based attendance system that utilizes RSS zoning to verify student presence, 
this approach requires students to install and interact with a mobile application to validate their location, translating in to a a volunteer-based architecture.


\section{Previous Works with SGX in Indoor Positioning}

Althoug few, some works using \ac{sgx} for indoor positioning have been found

% The following is regarding the article number as 20
% Privacy Protection in 5G Positioning and Location-based Services Based on SGX
To address the critical privacy concerns associated with collecting user location data, \citet{20_10.1145/3512892} proposed a trusted computing framework based on \ac{sgx} 
The authors argued that traditional privacy preserving techniques such as k-anonymity or homomorphic encryption, often incur prohibitive computational overheads or 
require reliance on a Trusted Third Party (TTP), which introduces a single point of failure.
In contrast, their work demonstrates that SGX enables a lightweight, integrated privacy scheme. 
By processing sensitive location data within a hardware protected enclave the system ensures that raw identity information remains inaccessible to the operating system or 
the service provider, allowing for secure positioning and occupancy analysis without exposing individual user trajectories. 
This approach validates the architectural choice in this thesis to utilize SGX enclaves for processing sensitive WiFi logs, ensuring compliance with privacy 
standards while maintaining system performance.

% The following is regarding the article number as 21
%Privacy-Preserving Location-Based Services by using Intel SGX
Complementing the architectural advantages mentioned in the previous work, \citet{21_Kulkarni:15319} provided an evaluation of \ac{sgx} for location based services, 
explicitly comparing it against traditional k-anonymity techniques. The study highlighted a trade-off in privacy handling, 
that traditional obfuscation methods degrades service accuracy to achieve privacy. 
In contrast, their experiments demonstrated that an \ac{sgx} based approach provides better result accuracy because the computation occurs on exact data within the secure 
enclave, protected from the host system.
Furthermore, the authors quantified the performance cost of this security, finding that \ac{sgx} introduces only a small compared to bare-metal implementation, due to 
the one time events such as the enclave creation, copying data, sealing and unsealing.
This finding is important for the current thesis, as it validates that shifting the occupancy estimation logic into an \ac{sgx} enclave is a viable strategy that secures 
student data without compromising the real time performance or the accuracy of the occupancy counts.

