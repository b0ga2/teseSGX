\chapter{Related Work}

This chapter provides an overview of the current state-of-the-art regarding indoor 
positioning and presence detection in indoor environments, also taking in account
how the privacy and security of the data was handled, if it was.

Driven by technological advancements and institutional needs, nearly all 
university buildings are now equipped with internet access provided through 
Wireless Access Points (APs). In the specific case study of the University of Aveiro 
\textcolor{red}{(mencionar departamento?)}, 
these APs allow for the determination of user location with a reasonable degree 
of precision. This location data can be utilized in various scenarios, including:

\begin{itemize}
    \item {\textcolor{red}{(Colocar referências)}}
    \item{\normalsize Building Occupancy}
    \item{\normalsize Energy Efficiency and Saving}
    \item{\normalsize People Counting}
    \item{\normalsize Statistics for Comparision to CO2 sensors}
    \item{\normalsize Monitoring Employee Performance}
    \item{\normalsize Enhancing wWrkplace Safety/Security}
    \item{\normalsize Optimize Space Utilization}
    \item{\normalsize Tracking Customer Behavior}
    \item{\normalsize Enabling Targeted Applications (p.e Smart Cleaning, Personalized Promotions)}
    \item{\normalsize Student Attendance Monitoring}
\end{itemize}

Although useful for generating interesting statistics, the utilization of original 
raw data necessitates access to sensitive information. 
This data is qualified as sensitive due to its high 
granularity and identifiability. Specifically, the raw logs contain:

\begin{itemize}
    \item {\normalsize Unique Identifier:} MAC addresses that act as static 
    hardware fingerprints for individuals.
    \item {\normalsize Universal User (UU):} The institutional email address 
    associated with the connection, providing a unique 
    reference to the individual user.
    \item {\normalsize Spatiotemporal Traces:} Precise records of where a user was 
    and for how long.
    \item {\normalsize Spatial Hierarchy:} The nested categorization of location 
    data (Country, City, Campus, Department, Floor), which allows for the 
    isolation of user movements within specific buildings or floors.
\end{itemize}

The aggregation of this data allows for "linkability", which is 
the ability to correlate a device's location history with external datasets 
(class schedules) thereby allowing the identification of users and 
exposing them to surveillance, profiling or stalking.

\section{Attacks on SGX}

Pode ser interessante mencionar alguns ataques conhecidos ao SGX