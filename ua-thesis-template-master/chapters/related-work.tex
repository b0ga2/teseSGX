\chapter{Related Work}
\label{chapter:related_work}

This chapter establishes the context for the proposed work by reviewing the state-of-the-art in 
Wi-Fi-based indoor positioning and occupancy detection. The review focuses not only on the technical 
methodologies for extracting location metrics but also on the associated privacy and security challenges.

Driven by technological advances and institutional needs, nearly all 
buildings are now equipped with internet access provided through 
\ac{ap}. These allow for the determination of user location with a reasonable degree 
of precision, due to the need of the user to authenticate to a specific AP 
in order to use the local network.

Although useful for generating interesting statistics, indoor positioning and presence detection
rely on the processing of sensitive information. 
This information is classified as sensitive due to its high level of detail 
and its potential to uniquely identify individuals.


The aggregation of this data creates a risk of linkability, allowing otherwise anonymous 
location traces to be cross-referenced with external records, such as class or work schedules. 
This process effectively de-anonymizes users, subjecting them to risks including surveillance 
and profiling, which constitutes a direct violation of their right to privacy.

FAZER DESCRIÇÂO DO QUE VAI SER ABORDADO E CONCLUSÂO (ver tese Duarte)

\section{Indoor Positioning System}

As defined by \citet{indoor_positioning_system} an Indoor Positioning System is viewed as 
the grouping of three components: positioning principles and corresponding algorithms; technologies; hardware equipment.
These are considered to have a meaningful impact on the performance of the positioning system (see Figure \ref{fig:struct_ips}).

\begin{figure}
  \includegraphics[width=\linewidth]{figs/indoor_positioning_system.jpg}
  \caption{General structure of the model of an indoor positioning system.}
  \label{fig:struct_ips}
\end{figure}


\subsection{Technologies used for Indoor Positioning System}

There are several technologies used for an Indoor Positioning System, \citet{indoor_positioning_system}
divided them in three categories, this overview has been expanded to include Optical 
and Environmental Sensors, reflecting the methodologies found in research literature,
as shown by Table \ref{table:ips_technologies_category}.

\begin{table}[htbp]
    \centering
    \caption{Indoor Positioning Categories}
    \label{table:ips_technologies_category}
    \begin{tabular}{ l  l }
        \hline
        \textbf{Category} & \textbf{Technologies} \\ \hline
        Radio Frequency (RF) & Wi-Fi \\
                             & Bluetooth Low Energy (BLE) \\
                             & Zigbee \\
                             & Ultra-wide band (UWB) \\
                             & Radio frequency identification (RFID) \\
                             & Indoor Global Navigation Satellite System (GNSS) \\
                             & Frequency modulation radio (FM-radio) \\ \hline
        Non-Electromagnetic Waves & Ultrasound \\
                                  & Geomagnetic Waves \\ \hline
        Full-spectrum light & Range imaging \\
                            & Visible light positioning \\
                            & Laser \\ \hline
        Optical & Cameras / Computer Vision \\
                & Infrared \\ \hline
        Environmental Sensors & CO2 Sensors \\ \hline
    \end{tabular}
\end{table}

\section {Wifi for Indoor Positioning}

Talvez falar que para a tese como só vai ser utilizado o Wifi as restantes tecnologias foram discartadas

Distinguir alguns termos como: positioning, location, etc

Mencionar usos do WIFI bla bla bla

There are several use cases of Wifi data to calculate for indoor positioning mentioned throughout the research literature, 
such: as Wifi probe technology done by \citet{23_WANG2018495},
\ac{rssi} performed by \citet{24_ZOU2017633} and \ac{csi} researched by \citet{25_8293759}.

\subsection{Building Occupancy}

01

Estimating building occupancy is a necessary step for optimizing building operations, in this scenario
WiFi connection counts are presented as a cost-efficient proxy for occupancy when comparing to most commom methods, 
even showing a strong correlation with actual people counts as shown by \citet{22_8913341}.

In 2020, \citet{01_ALISHAHI2021107936} proposed a framework that used machine learning models and 
statistical analysis methods to predict patterns of building occupancy based on Wifi connections.

The authors identified that more traditional methods, often suffer from latency and high maintenance costs. 
Instead, they utilized associated WiFi device counts, which refer to all devices trying, not necessarily being able to, 
authenticate with the \ac{ap}.

Their methodology focused on three main objectives regarding occupancy:
\begin{enumerate}
    \item \textbf{Occupancy Pattern:} This goal was divided in two phases, the first one was the identification of groups that
                                      similar patterns and the second was to develop prediction models for each group.
    \item \textbf{Peak Occupancy:} Analyze and predict peak occupancy by couting the number of maximum WiFi connections and the 
                                   time of its occurence from the previous week.
    \item \textbf{Arrival and Departure Times:} Mapping of the \ac{ap} location to the associated zone, then through the observation
                                                of the first and last WiFi connection calculate arrival and departure.
\end{enumerate}

The implementation of this framework demonstrated positive results, achieving an 
average prediction accuracy ($R^{2}$) of 0.98 for weekdays and 0.81 for weekends.
However, \citet{01_ALISHAHI2021107936} also connoted limitations that must be acknowledged, although
the correlation between connection counts and occupancy is significant,
the prediction models require calibration with ground-truth data to account for stationary devices 
such as printers and desktops, and that the variance in the number of devices carried by each occupant should be taken in account.
This becomes more pronounced at smaller spaces, such as room-level or zone-level analysis, 
where attributing connections to a specific zone is difficult without access to granular connection details like the
ones given by \ac{rssi}.



02

\subsection{Room Occupancy}

07

\subsection{Energy Efficiency and Saving}

03

16

\subsection{People Counting}

02


\subsection{Statistics for Comparision with Other Sensors}

04

09

\subsection{Monitoring Employee Performance}

05

\subsection{Enhancing wWrkplace Safety/Security}

05

\subsection{Optimize Space Utilization}

05

\subsection{Tracking User Behavior}

11

\subsection{Enabling Targeted Applications (p.e Smart Cleaning, Personalized Promotions)}
\subsection{Student Attendance Monitoring}

06

07

13

15

\subsection{Navigation}

18

\subsection{Development of Indoor Positioning Method}

19









\section{Dealing with User Privacy}

Alguns não usam como o (02) porque 
neste caso é um radar de sinais

\subsection{Avoids Collecting Sensitive Data}

01

\subsection{Doesn't Deal with it}

03

05

09

17

18

19

\subsection{}

04

\subsection{Anonymized Data}

06

\subsection{Ethical Clearance}

07

\subsection{Volunteer-Based Model or Opt-in Model}

11

13

15

