\chapter{Related Work}
\label{chapter:related_work}

This chapter establishes the context for the proposed work by reviewing the state-of-the-art in 
WiFi-based indoor positioning and occupancy detection. The review focuses not only on the technical 
methodologies for extracting location metrics but also on the associated privacy and security challenges.

Driven by technological advances and institutional needs, nearly all 
buildings are now equipped with internet access provided through 
\ac{ap}. These allow for the determination of user location with a reasonable degree 
of precision, due to the need of the user to authenticate to a specific AP 
in order to use the local network.

Although useful for generating interesting statistics, indoor positioning and presence detection
rely on the processing of sensitive information. 
This information is classified as sensitive due to its high level of detail 
and its potential to uniquely identify individuals.


The aggregation of this data creates a risk of linkability, allowing otherwise anonymous 
location traces to be cross-referenced with external records, such as class or work schedules. 
This process effectively de-anonymizes users, subjecting them to risks including surveillance 
and profiling, which constitutes a direct violation of their right to privacy.

FAZER DESCRIÇÂO DO QUE VAI SER ABORDADO E CONCLUSÂO (ver tese Duarte)

\section{Indoor Positioning System}

As defined by \citet{indoor_positioning_system} an Indoor Positioning System is viewed as 
the grouping of three components: positioning principles and corresponding algorithms; technologies; hardware equipment.
These are considered to have a meaningful impact on the performance of the positioning system (see Figure \ref{fig:struct_ips}).

\begin{figure}
  \includegraphics[width=\linewidth]{figs/indoor_positioning_system.jpg}
  \caption{General structure of the model of an indoor positioning system.}
  \label{fig:struct_ips}
\end{figure}


\subsection{Technologies used for Indoor Positioning System}

There are several technologies used for an Indoor Positioning System, \citet{indoor_positioning_system}
divided them in three categories, this overview has been expanded to include Optical 
and Environmental Sensors, reflecting the methodologies found in research literature,
as shown by Table \ref{table:ips_technologies_category}.

\begin{table}[htbp]
    \centering
    \caption{Indoor Positioning Categories}
    \label{table:ips_technologies_category}
    \begin{tabular}{ l  l }
        \hline
        \textbf{Category} & \textbf{Technologies} \\ \hline
        Radio Frequency (RF) & WiFi \\
                             & Bluetooth Low Energy (BLE) \\
                             & Zigbee \\
                             & Ultra-wide band (UWB) \\
                             & Radio frequency identification (RFID) \\
                             & Indoor Global Navigation Satellite System (GNSS) \\
                             & Frequency modulation radio (FM-radio) \\ \hline
        Non-Electromagnetic Waves & Ultrasound \\
                                  & Geomagnetic Waves \\ \hline
        Full-spectrum light & Range imaging \\
                            & Visible light positioning \\
                            & Laser \\ \hline
        Optical & Cameras / Computer Vision \\
                & Infrared \\ \hline
        Environmental Sensors & CO2 Sensors \\ \hline
    \end{tabular}
\end{table}

\section {Wifi for Indoor Positioning}

Although there are several technologies used for indoor positioning system, we will focus
on Wifi, since the present use case is based on Wifi.

In the researched literature we came across several terms related with indoor positioning, such as
positioning, location, detection, counting and tracking. MELHORAR ESTA PARTE e citar o 24_ZOU2017633
e dizer que para o meu caso olhamos para as coisas de uma maneira mais "broad" e sem tanto detalhe
para o uso especifico

Fazer ponto bonita para alguns exemplos

There are several use cases of Wifi data to calculate for indoor positioning mentioned throughout the research literature, 
such: as Wifi probe technology done by \citet{23_WANG2018495},
\ac{rssi} performed by \citet{24_ZOU2017633}, \ac{csi} researched by \citet{25_8293759}
and many other that will mentioned throughout the present chapter.

\subsection{Building Occupancy}

Estimating building occupancy is a necessary step for optimizing building operations, in this scenario
WiFi connection counts are presented as a cost-efficient proxy for occupancy when comparing to most commom methods, 
even showing a strong correlation with actual people counts as shown by \citet{22_8913341}.

% The following is regarding the article number as 01
% 

In 2020, \citet{01_ALISHAHI2021107936} proposed a framework that used machine learning models and 
statistical analysis methods to predict patterns of building occupancy based on Wifi connections.

The authors identified that more traditional methods, often suffer from latency and high maintenance costs. 
Instead, they utilized associated WiFi device counts, which refer to all devices trying, not necessarily being able to, 
authenticate with the \ac{ap}.


The implementation of this framework demonstrated positive results, achieving an 
average prediction accuracy ($R^{2}$) of 0.98 for weekdays and 0.81 for weekends.
However, \citet{01_ALISHAHI2021107936} also connoted limitations that must be acknowledged, although
the correlation between connection counts and occupancy is significant,
the prediction models require calibration with ground-truth data to account for stationary devices 
such as printers and desktops, and that the variance in the number of devices carried by each occupant should be taken in account.
This becomes more pronounced at smaller spaces, such as room-level or zone-level analysis, 
where attributing connections to a specific zone is difficult without access to granular connection details like the
ones given by \ac{rssi}.


\subsection{Room Occupancy}

07

\subsection{Energy Efficiency and Saving}

% The following is regarding the article number as 03
% WiFi based occupancy detection in a complex indoor space under discontinuous wireless communication: A robust filtering based on event-triggered updating

\citet{03_WANG2019228} addressed the challenge of indoor positioning and discontinuous WiFi 
communication specifically by smartphones, a phenomenon where devices suspend data transmission 
to conserve battery, causing users to disappear from conventional scanning systems. 
To overcome this, the authors developed an event-triggered updating method that estimates occupancy 
based on entering and exiting events instead of relying on continuous signal detection, 
with these events, the system estimates the device owner's location using \ac{rssi} data. 

Furthermore, the framework integrates a location filter using \ac{rssi} thresholds to discard devices 
located outside the target zone by applying a study of the mean values of the measured data in inside 
and outside positions, and a non-human filter to exclude stationary equipment, 
such as printers, via \ac{mac} address analysis.

However, the authors acknowledge significant limitations. First, the method is tailored to the specific 
WiFi operation patterns of smartphones and may not generalize to devices with different behaviors. 
Second, the reliance on \ac{rssi} thresholds restricts the system to zone-level accuracy, in a 
virtually partitioned zones without physical walls, outside devices closer to detection nodes may be 
miscalculated as inside, and thirdly the detection errors arise from user behavior, 
such as occupants carrying multiple devices or separating from their smartphones, which the authors 
suggest requires data fusion with other sensors to resolve.


16

\subsection{People Counting}

% The following is regarding the article number as 02
% Occupancy Detection and People Counting Using WiFi Passive Radar

In the 2020 \ac{ieee} Radar Conference \citet{02_9266493} presented a methodology 
for people counting based on \ac{pwr}. The authors presented this approach 
to overcome the limitations of standard WiFi sensing, since \ac{rssi} methods are 
prone to unpredictable fluctuations and false positives due to multipath effects, 
and \ac{csi} techniques typically require specific hardware modifications 
or high-rate transmissions that degrade network throughput.

The overall results presented a high accuracy of 99.54\% for 
tasks of occupancy detection and 98.80\% for people counting. 
Taking into account the positive results, the authors defend that the \ac{pwr} 
system is applicable as it leverages existing commercial WiFi 
\ac{ap}s without requiring any modifications to the WiFi infrastructure or 
additional devices on the network. 
However, this advantage comes with added computational 
complexity, as the system relies completely on signal processing.


\subsection{Statistics for Comparision with Other Sensors}

% The following is regarding the article number as 04
% Effectiveness of using WiFi technologies to detect and predict building occupancy

In 2017, \citet{04_OUF2017005} presented a case study to demonstrate the effectiveness of using 
WiFi to detect occupancy as opposed to the more common $CO_2$ sensors.
To facilitate this comparison, the authors simultaneously monitored WiFi connection counts and 
$CO_2$ concentration levels in a university classroom over one week, using manual occupant counts 
as ground truth.

The analysis revealed that WiFi counts served as a superior predictor of occupancy, 
exhibiting a stronger statistical correlation ($r=0.839$) than $CO_2$ levels ($r=0.728$).
Furthermore, the authors highlighted that while $CO_2$ sensors suffered from a detection lag of 
approximately 20 minutes and susceptibility to non-occupancy related fluctuations, WiFi data 
provided a more accurate, real-time proxy for occupancy with the added benefit of utilizing existing 
infrastructure without additional cost.

09

\subsection{Monitoring Employee Performance}

05
% The following is regarding the article number as 05
% Effectiveness of using WiFi technologies to detect and predict building occupancy

\subsection{Enhancing wWrkplace Safety/Security}

05

\subsection{Optimize Space Utilization}

05

\subsection{Tracking User Behavior}

11

\subsection{Enabling Targeted Applications (p.e Smart Cleaning, Personalized Promotions)}
\subsection{Student Attendance Monitoring}

06

07

13

15

\subsection{Navigation}

18

\subsection{Development of Indoor Positioning Method}

19









\section{Dealing with User Privacy}

Alguns não usam como o (02) porque 
neste caso é um radar de sinais

\subsection{Avoids Collecting Sensitive Data}

01

\subsection{Doesn't Deal with it}

03

05

09

17

18

19

\subsection{}

04

\subsection{Anonymized Data}

06

\subsection{Ethical Clearance}

07

\subsection{Volunteer-Based Model or Opt-in Model}

11

13

15

