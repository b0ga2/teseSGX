\chapter{Context}

Intro: parágrafo pequeno a explicar que este capítulo introduz as TEE

\section{Trusted Execution Environments (TEEs)}
Definição de TEE (isolamento de código e dados).

Propriedades principais: Confidencialidade e Integridade.

Diferença entre o ambiente não confiavel e o TEE.

\section{Intel Software Guard Extensions (SGX)}
    \subsection{Architecture and Workflow}
    Como funcemina:

    Explicar o conceito de Enclaves.

    Explicar a divisão da memória (EPC - Enclave Page Cache).

    Mecanismo de ECalls e OCalls (entrar e sair do enclave e comunicação com o mundo "inseguro").

    \subsection{Security Mechanisms: Attestation and Sealing}
    
    Remote Attestation: Como provar que o código é legítimo 

    Sealing: Como guardar dados cifrados no disco.

    \subsection{Relevance to Privacy and Preserving Analytics}
    Explicar por que razão o SGX é bom para processar dados sensíveis

\section{Alternative TEE Technologies}
ARM TrustZone

AMD SEV

Maybe justificar porquê o SGX, se bem que estava no titulo da tese, não escolhi nada

\section{Technical Challenges: Memory Limitations in SGX}

    \subsection{The Enclave Page Cache (EPC) Bottleneck}
    Explicar que a memória segura é limitada, ver slides para mais detalhes


