\chapter{Context}
\label{chapter:context}

The present chapter provides a comprehensive understanding of the technical underpinnings of 
\ac{sgx} and the protection of sensitive data.
It begins with the introduction of the concept of sensitive data, followed by \ac{tee}s, giving an overview of this type of mechanism. 
Then the chapter compares technologies, analyzing the distinctions between several \ac{tee}s implementations. 
After this, the discussion focuses on Intel \ac{sgx}, detailing its architecture, workflow, and security mechanisms, followed by
the technical challenges of data processing in an enclave and concluding with potential attacks on \ac{sgx}.

\section{Sensitive Data}

Driven by technological advances and institutional needs, nearly all
buildings are now equipped with internet access provided through 
\ac{ap}. These allow for the determination of user location with a reasonable degree 
of precision, due to the need of the user to connect to a specific AP 
in order to use the network.

Although often categorized as technical metadata, it presents privacy challenges when analyzed by definition of sensitive data, 
while raw location logs may not possess 
intrinsic sensitiveness, like a direct medical diagnosis might, its possible to acquire sensitive information through their 
processing context. As \citet{04_article} argue, the critical factor is the computational distance required to infer sensitive 
information from apparently neutral data.

In the case of indoor positioning, this computational distance is low, the aggregation of location traces creates a high 
risk of linkability, where disparate datasets, such as anonymous WiFi logs and public class schedules, can be easily cross referenced. 
This reduces the effort necessary to de-anonymize users, effectively transforming technical data into personal 
identifiable information.

Example of sensitive information used for indoor location mentioned throughout this document:

\begin{itemize}
    \item \textbf{\ac{mac} Address:} A persistent hardware identifier assigned to the network interface controller, 
    this serves as a static fingerprint that uniquely identifies a user's device across different sessions and locations.
    \item \textbf{\ac{uu}:} The unique user identifier (such as a student ID, employee username, or email) 
    required for network authentication.
    \item \textbf{Connection Logs:} Time records containing precise association and disassociation timestamps,
    these allow for the reconstruction of a user's daily routine, duration of stay, and punctuality patterns.
    \item \textbf{Spatial Hierarchy:} Topological data classifying location from macro to micro levels.
    \item \textbf{Visual and Biometric Data:} In systems utilizing optical sensors, raw data includes high-resolution images 
    or video feeds capable of revealing biometric features and behavioral signatures.
\end{itemize}

\section{Trusted Execution Environments (TEEs)}
% Definir Trusted Computing e talvez exemplo
Before defining \ac{tee}s, it is essential to understand the underlying concept of \ac{tc}, as defined by 
\citet{01_7345265}, a \ac{tc} aims to enable systems to achieve secure computation, privacy, and data protection. 
Historically, \ac{tc} relied on a dedicated hardware component to provide a functional interface for security operations, 
one common example is the \ac{tpm}, which allows a system to prove its integrity and securely store cryptographic keys,  
however, a limitation of the \ac{tpm} is its inability to perform isolated execution of arbitrary code, 
since it is mainly designed for passive storage and reporting, rather than active processing.
But more recent approaches to \ac{tc}, shift the execution of arbitrary code to a trusted environment that allows tamper resistant 
execution of its applications.

%Definição de TEE (isolamento de código e dados).
%Propriedades principais: Confidencialidade e Integridade.
While various terminologies exist in the literature, this dissertation adopts the term \ac{tee}, this is defined by \citet{01_7345265} 
as a secure, integrity protected computational environment, typically with isolated memory and specific storage capabilities. 
Its primary objectives are to guarantee the authenticity of the executed code, the integrity of the runtime state and the confidentiality of the data, 
like \ac{cpu} registers, while effectively protecting code, memory, and sensitive \ac{io} from unauthorized access, in addition 
a \ac{tee} should have the ability to support remote attestation, allowing third parties to verify the trustworthiness of the environment, 
although specific implementation details may vary across different types of \ac{tee}s. 

% Distinguir entre hardware e software
The implementation of these can differ, leading to a categorization of \ac{tee}s based on their enforcement mechanism, 
hardware-based \ac{tee}s rely on specific hardware extensions, such as Intel \ac{sgx}, to enforce isolation at the lower level, 
effectively anchoring trust in the physical processor, in contrast, software-based \ac{tee}s depend on a privileged software layer 
typically a hypervisor or microkernel to protect the execution environment. 
\citet{01_7345265} proposed some definitions represented in the following figure.
\begin{figure}
  \includegraphics[width=\linewidth]{figs/TEE_Building_Blocks.png}
  \caption{Overview of \ac{tee} building blocks, image by \citet{01_7345265}.}
  \label{fig:TEE_Building_Blocks}
\end{figure}
\newpage
As seen in the previous picture, a TEE in composed by building blocks such as:
\begin{itemize}
    \item{\normalsize Secure Boot: guarantees that only code of a certain property can be loaded during boot.}
    \item{\normalsize Secure Scheduling: provides a balanced coordination between the \ac{tee} and the rest of the system.}
    \item{\normalsize Inter-Environment Communication: Interface that allows the \ac{tee} to communicate with the rest of the system.}
    \item{\normalsize Secure Storage: Storage where confidentiality, integrity and authorized access of stored data is guaranteed.}
    \item{\normalsize Trusted I/O Path: Communication between \ac{io} peripherals with authenticity and sometimes confidentiality.}
\end{itemize}

% Diferença entre o ambiente não confiavel e o TEE.
In contrast to the isolation of the \ac{tee}, the remaining portion of the system is often referred to as the \ac{ree} or the untrusted environment, 
this includes the traditional operating system, the hypervisor, and the standard application stack (yellow component of the Figure\ref{fig:ree_vs_tee}). 
While the \ac{ree} is optimized for performance, versatility, and accessing peripherals, it looses the security guarantees 
required for sensitive data processing, as it makes it vulnerable software attacks. 
The architectural principle of trusted computing is the strict separation between these two worlds, 
the \ac{tee} protects its assets even when the surrounding \ac{ree} is fully compromised.
This separation can be observed in the Figure \ref{fig:ree_vs_tee}.
\begin{figure}
  \begin{center}
  \includegraphics[scale=0.7]{figs/ree_vs_tee.png}
  \caption{Overview of \ac{ree} and of \ac{tee}, image by \citet{02_phdthesis}.}
  \label{fig:ree_vs_tee}
  \end{center}
\end{figure}

\section{TEE Technologies}
Several \ac{tee} implementations exist, such as ARM TrustZone, AMD \ac{sev}, and Intel most recent \ac{tdx}, 
this chapter provides a comparative overview of these technologies, however, Intel \ac{sgx}, the focus of this dissertation, 
is reviewed in greater detail in the subsequent chapter.

\subsection{ARM TrustZone}

% All this text was based on Demystifying Arm TrustZone: A Comprehensive Survey
Initially introduced in 2004, ARM TrustZone is a set of hardware security extensions available on ARM processors, while first 
implemented on the Cortex-A series, it was later adapted for the more recent Cortex-M series, with some architectural differences 
suited for microcontrollers. 
Contrary to a standalone component, TrustZone is an approach designed to be integrated into a \ac{soc}, 
securing not just the processor but the entire electronic device, as described by \citet{05_arm_trustzone}, in this model
the architecture is centered around the concept of distinct domains, the Secure World, the trusted environment, and 
the Normal World, the untrusted environment, as seen in Figure \ref{fig:arm_trustzone}.

\begin{figure}
  \begin{center}
  \includegraphics[scale=0.9]{figs/TrustZone_architecture.png}
  \caption{ARM TrustZone technology, image by \citet{05_arm_trustzone}.}
  \label{fig:arm_trustzone}
  \end{center}
\end{figure}

\newpage

This isolation allows software to execute with different privilege levels, preventing untrusted applications 
in the Normal World from directly accessing resources allocated to the Secure World. 
The state of the processor is determined by the \ac{ns} bit, typically located in the \ac{scr}, which effectively partitions the system's hardware resources.

To manage the transition between worlds, TrustZone employs a monitor mode, which executes in a secure state 
independent of the \ac{ns} bit, this mode bridges the two software stacks and is entered via the privileged \ac{smc} instruction or through specifically 
configured hardware exceptions. 

Hardware isolation is reinforced by banked registers, which guarantees that security critical system registers are 
either hidden from the untrusted environment or are strictly accessed from the trusted environment.
Regarding memory infrastructure, TrustZone introduces components such as the \ac{tzasc} for \ac{dram} and the \ac{tzma} for 
off-chip \ac{rom}/\ac{sram}, these controllers partition memory into trusted and untrusted regions, preventing the Normal World from accessing secure 
memory while allowing the Secure World full access. 
TrustZone \ac{mmu} maintains separate virtual to physical translation tables for each domain, and cache isolation is 
enforced via an additional tag bit on cache lines. 
Finally, the \ac{tzpc} extends this isolation to system peripherals, restricting access to specific devices based 
on the security domain.

In recent years, academic and industrial interest in this technology has surged, driven by the availability of 
technical documentation and accessible development platforms from manufacturers like Xilinx, this transparency has facilitated 
adoption within the community, particularly in the context of \ac{rd}.

\subsection{AMD SEV}

Introduced in 2016, \ac{sev} is stated to be the first x86 technology that was design to isolate \ac{vm}s from the hypervisor, since 
historically these have been considered trusted components, this allows for example, clients in the cloud to secure their \ac{vm} 
workload from the cloud administrator, keeping their data confidential and minimizing exposure.
AMD did this using main memory encryption in \ac{sev}, with this, individuals \ac{vm}s were assigned an individual key to encrypt all data, 
preventing the hypervisor from reading clear text information, as cited by \citet{05_amd_sev}.
Later in 2017, a new version came out, the AMD \ac{sev-es} that added the additional feature of \ac{cpu} register state protection, in order to 
do this, in \ac{sev-es} the \ac{vm} register state is encrypted at each hypervisor transition, preventing the hypervisor to see 
the data that is being used, adding confidentiality to in-memory data.

The most recent version is the \ac{sev-snp}, built on top of the previous versions while adds hardware based protections like 
memory integrity protection to prevent attacks like data replay, memory re-mapping, it also implements 
protection in interrup behavior. 

This is, in the reviewed literature, according to \citet{06_amdsev_inteltdx}, the first \ac{cvm}, 
which are \ac{vm}s made to to alleviate the programmability and usability challenges of the previously proposed 
trusted computing technologies, by providing abstraction at the level of the \ac{vm}.
This technology uses a dedicated processor, the \ac{asp} to manage the security features, this is an ARM-based processor that 
is separated from the main 32-bit core, but directly integrated in the \ac{cpu}, which also allows for remote attestation.

\begin{figure}
  \begin{center}
  \includegraphics[scale=1]{figs/amd_sev_snp_architecture.png}
  \caption{AMD \ac{sev} Architecture, image by \citet{06_amdsev_inteltdx}.}
  \label{fig:amd_sev}
  \end{center}
\end{figure}

\newpage

The Figure \ref{fig:amd_sev} shows the overview of the AMD \ac{sev-snp}, where green zones ilustrate \ac{cvm}s with unmodified software (orange) 
and with modified software (purple) and the blue areas are trusted components, the vertical black line indicates that 
each \ac{cvm} is isolated from the host and other encrypted \ac{vm}s. This \ac{vm} executes in an isolated 
environment where its memory and state registers are encrypted via a unique key managed by the \ac{asp}, 
this safeguards the integrity of guest address translations and strictly isolates the 
execution by blocking \ac{smi} events and preventing access from the \ac{smm}. 
This architecture also introduces \ac{vmpl}, a four-tiered privilege hierarchy to traditional x86 rings  
that enables a \ac{svsm} running at the highest level 
\ac{vmpl}0 to securely offer services like virtual \ac{tpm} and live migration.

\subsection{Intel TDX}

With the same goals as AMD \ac{sev}, Intel \ac{tdx} is an extension introduced with the 4th Generation Intel Xeon Scalable processors to 
enable confidential computing. 
This technology facilitates the deployment of \ac{vm}s, referred to as \ac{td}s, under the supervision of the \ac{seam}, this provides encrypted \ac{cpu} 
state and memory, with integrity protection and remote attestation, addressing scenarios where the cloud host platform is considered 
untrustworthy within the threat model
\footnote{A threat model is a structured approach to identifyingpotential security risks, defining attacks from the perspective of 
an adversary to strengthen system defenses \citep{threat_model}.}. 

Technically, \ac{tdx} builds upon other Intel technologies such as \ac{vt} and \ac{mktme}, together, these components enforce 
the confidentiality and integrity of \ac{td} memory and \ac{cpu} states, ensuring that they can neither be accessed nor tampered with by other 
entities in the environment, such as the hypervisor or neighboring \ac{vm}s in the same cloud tenant. Although it also uses \ac{dcap} and \ac{sgx} 
for remote attestation. 

According to \citet{12_intel_tdx} it achieves the previous using a combination of three components, namely, memory access control, runtime memory encryption and an 
Intel-signed \ac{tdx} module that does the handling of security-sensitive \ac{td} management operations. 

\newpage
\begin{figure}
  \begin{center}
  \includegraphics[scale=0.5]{figs/intel_tdx_architecture.png}
  \caption{Intel \ac{tdx} Architecture, image by \citet{12_intel_tdx}.}
  \label{fig:intel_tdx}
  \end{center}
\end{figure}

The Figure \ref{fig:intel_tdx} illustrates the runtime architecture of the \ac{tdx} technology. The two main components are the \ac{tdx} compatible processors, that provide 
functional capabilities like hardware-assisted virtualization, memory encryption, integrity protection, and the \ac{tdx} Module, that facilitates the manipulation of 
\ac{td} while applying security guarantees, besides this, the module provides two interface functions, a host-side interface and a guest-side interface. 
The previous module is loaded in to the \ac{seam} range, a portion of memory that is reserved during boot. Another component present in this range 
is the \ac{seam} loader, which is responsible for installing and updating the module. 
\ac{seam} provides two execution modes, the \ac{seam} \ac{vmx} root mode and non-root mode, a \ac{tdx} compatible hypervisor operates in the \ac{vmx} root mode 
and uses the SEAMCALL introduction to call host-side functions, functions that start with TDH, of the modulea. 

This is possible since the the logical processor changes \ac{vmx} root mode into \ac{seam} \ac{vmx} root mode and starts executing code within the TDX Module, 
upon completing it returns in \ac{vmx} root mode with the instruction SEAMRET. 
The \ac{td}s run in \ac{seam} \ac{vmx} non-root mode, since \ac{tdx} supports the execution of unmodified user-level applications within a \ac{td}, like a normal 
\ac{vm}, but the guest \ac{os} kernel (TDX-enlightened OS in Figure \ref{fig:intel_tdx}), needs to be altered in order to align with the \ac{tdx} plataform, namely 
managing \ac{tdx} exceptions via an internal handler, implementation of a mechanism for communication between the \ac{td} and the \ac{tdx} module, 
transitioning memory pages from private to shared \ac{io} operations, although implementation details may vary according to the \ac{os} in use.


\section{Intel Software Guard Extensions (SGX)}

Maybe justificar porquê o SGX, se bem que estava no titulo da tese, não escolhi nada
  Usar artigo de comparação com a AMD, e dizer que no TDX não existe a necessidade

    \subsection{Architecture and Workflow}
    Como funcemina:

    Explicar o conceito de Enclaves.

    Explicar a divisão da memória (EPC - Enclave Page Cache).

    Mecanismo de ECalls e OCalls (entrar e sair do enclave e comunicação com o mundo "inseguro").

    \subsection{Security Mechanisms: Attestation and Sealing}
    
    Remote Attestation: Como provar que o código é legítimo 

    Sealing: Como guardar dados cifrados no disco 
    Unsealing

    \subsection{Relevance to Privacy and Preserving Analytics}
    Explicar por que razão o SGX é bom para processar dados sensíveis

\section{Technical Challenges: Memory Limitations in SGX}

    \subsection{The Enclave Page Cache (EPC) Bottleneck}
    Explicar que a memória segura é limitada, ver slides para mais detalhes
    Mencionar algumas abordagens no related work


\section{Attacks on SGX}

Pode ser interessante mencionar alguns ataques conhecidos ao SGX
Não sei se isto deveria estar aqui ou no 



